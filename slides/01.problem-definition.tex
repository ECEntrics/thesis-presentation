\section{Ορισμός του προβλήματος}
\begin{frame}
	\frametitle{Ορισμός του προβλήματος}
	\vspace{-\baselineskip}

	Προβλήματα στις σύγχρονες εφαρμογές αρχιτεκτονικής client-server:
	\begin{itemize}
		\item Αδύναμη ασφάλεια η οποία δεν αποτελεί προτεραιότητα
		\item Καμία εγγύηση διαθεσιμότητας
		\item Απαίτηση εμπιστοσύνης προς τον εξυπηρετητή
		\item Μη αυθεντικοποίηση των δεδομένων
		\item Έλλειψη ή/και καταπάτηση ελευθερίας του λόγου
	\end{itemize}
\end{frame}

\note{
	Από τις συζητήσεις μας με τον κύριο Δημάκη, αλλά και από τις εμπειρίες μας στον σύγχρονο διαδικτυακό διάλογο, έγινε σαφής η αδυναμία των κεντροποιημένων πλατφόρμων επικοινωνίας να αποδώσουν ορισμένα χαρακτηριστικά που είναι βασικά για την ελευθερία και την ίση συμμετοχή. Αναγνωρίσαμε ότι, μερίδα των προβλημάτων προκύπτει λόγω της σχέσης πελάτη-εξυπηρετητή που επικρατεί στις πλατφόρμες αλλά και των προτεραιοτήτων που θέτουν οι εταιρίες πίσω από αυτές.

	Συγκεκριμένα, η ασφάλεια είναι συχνά στόχος χαμηλής προτεραιότητας. Η διαθεσιμότητα πλήττεται σημαντικά καθώς ο εξυπηρετητής μπορεί ανά πάσα στιγμή να χάσει τα δεδομένα ή να αρνηθεί την πρόσβαση σε αυτά. Όλη η αρχή λειτουργίας των κεντροποιημένων αυτών συστημάτων απαιτεί την εμπιστοσύνη από τους πελάτες προς τους εξυπηρετητές, ενώ επίσης ελάχιστες είναι οι πλατφόρμες που επιτρέπουν την αυθεντικοποίηση των δεδομένων από τους χρήστες, (κάτι που τεχνολογικά είναι αρκετά εύκολο). (η αυθεντικοποίηση αφορά τον έλεγχο από τον παραλήπτη ότι ένα μήνυμα είναι γνήσιο και προέρχεται πράγματι από τον αποστολέα)

	Τέλος, ο εξυπηρετητής διατηρεί τον πλήρη έλεγχο της συμμετοχής στον διάλογο και μπορεί να αποκλείσει οποιονδήποτε χρήστη ή οντότητα από αυτόν.
}
