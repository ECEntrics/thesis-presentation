\section{Ανοιχτά Θέματα}
\begin{frame}
	\frametitle{Ανοιχτά Θέματα}
	\begin{itemize}
		\item Διαχείριση των τελών του Ethereum
		\item Διανομή των Ethereum token
		\item Εναλλακτικά συστήματα ψηφοφορίας
		\item Συστήματα απόδοσης εμπιστοσύνης
	\end{itemize}
\end{frame}

\note{
	Έτσι, φτάνουμε στα ανοιχτά θέματα.
	
	Το πρώτο και σημαντικότερο είναι η διαχείριση των τελών στις συναλλαγές του Ethereum. Αυτό που προτείνουμε και είναι γενικά η τάση που επικρατεί είναι η χρήση μετασυναλλαγών. Στην περίπτωσή μας αυτό θα σήμαινε π.χ. ότι η πληρωμή των τελών θα προωθείται από τον χρήστη στην κοινότητα που ανήκει.
	
	Ένα δεύτερο ζήτημα είναι η δημιουργία μηχανισμών για την ανώνυμη διανομή των token στους χρήστες. Αν και στην υλοποίησή μας αφήσαμε ανοιχτή τη διαδικασία διανομής των token στην εκάστοτε κοινότητα, θα πρέπει να τονίσουμε ότι σχετικά πρόσφατα άρχισαν να εμφανίζονται υλοποιήσεις στο Ethereum που μπορούν να ενσωματωθούν και που την απλοποιούν κατά πολύ, επιτρέποντας την ανώνυμη μετακίνηση token από διεύθυνση σε διεύθυνση.
	
	Επίσης, οι διαδικασίες των ψηφοφοριών μπορούν να επεκταθούν και να ορίζονται αυθαίρετα από την κάθε κοινότητα. Για παράδειγμα να υπάρχει επιλογή για ψηφοφορία με σειρά προτίμησης ή για έμμεση ψηφοφορία.
	
	Τέλος, μπορούν να χτιστούν συστήματα απόδοσης εμπιστοσύνης για τους χρήστες, τα οποία να λειτουργούν κυρίως συμπληρωματικά με τα θέματα των τελών και των ψηφοφοριών. Για παράδειγμα σε μία κοινότητα ένας χρήστης να απαλλάσσεται από τέλη εάν περάσει ένα όριο εμπιστοσύνης ή η ισχύς της ψήφου του να ορίζεται από αυτήν.
}
