\section{IPFS}
\begin{frame}
	\frametitle{IPFS}
	\vspace{-\baselineskip}
	\begin{center}
		\includegraphics[width=.1\paperwidth]{assets/figures/ipfs-logo}
	\end{center}
	\vspace{.5\baselineskip}
	IPFS - κατανεμημένο σύστημα αποθήκευσης
	\begin{itemize}
		\item Δίκτυο ομότιμων κόμβων (P2P)
		\item Διευθυνσιοδότηση περιεχομένου (content addressing)
		\item Καρφίτσωμα δεδομένων (pinning)
	\end{itemize}
\end{frame}

\note{
	Συνεχίζουμε με το τρίτο επίπεδο της στοίβας. Όπως είπαμε, για την αποθήκευση του κύριου όγκου των δεδομένων επιλέχθηκε το IPFS. Αυτό έγινε γιατί η αποθήκευση δεδομένων επί του blockchain μεταφράζεται πάντα σε επιπλέον κόστη συναλλαγών, τα οποία στην περίπτωσή μας θα ήταν απαγορευτικά μεγάλα. 
	
	Περιγράφοντας το IPFS ΠΟΛΥ απλοϊκά, μπορούμε να το παρομοιάσουμε με το Bittorrent. Πρόκειται δηλαδή για ένα κατανεμημένο σύστημα ομότιμων κόμβων, το οποίο αποθηκεύει και διαμοιράζει δεδομένα.
	
	Εδώ το περιεχόμενο δεν προσδιορίζεται από την τοποθεσία (π.χ. HTTP), άλλα από το τι περιλαμβάνει. Έχουμε δηλαδή διευθυνσιοδότηση του περιεχομένου (content addressing), κάθε κομμάτι του οποίου αποκτά ένα μοναδικό αναγνωριστικό, ένα content id.
	
	Τέλος θα πρέπει να σημειώσουμε πως οι κόμβοι που διαθέτουν και διαμοιράζουν το περιεχόμενο αντιμετωπίζουν, ως προεπιλογή, τα αποθηκευμένα δεδομένα ως προσωρινή μνήμη. Για την αποφυγή της διαγραφής τους τα δεδομένα θα πρέπει να καρφιτσώνονται, δηλαδή να γίνονται pinned από τους κόμβους που επιθυμούν να τα συνεχίσουν να τα διατηρούν. 
}