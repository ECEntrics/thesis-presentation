\section{Ethereum}
{
	\begin{frame}
		\frametitle{Ethereum}
		\vspace{-\baselineskip}
		\begin{center}
			\includegraphics[width=.1\paperwidth]{assets/figures/ethereum-logo}
		\end{center}
		Ethereum - μία προγραμματιστική πλατφόρμα βασισμένη στο blockchain
		\begin{itemize}
			\item Έξυπνα συμβόλαια = αυτόνομα κομμάτια κώδικα επί του blockchain
			\item DApps (Decentralized Applications) = smart contracts + user interfaces
			\item Tokens (υπονομίσματα)
		\end{itemize}
	\end{frame}
}

\note{
	Έτσι, περνάμε στο Ethereum. Το Ethereum είναι ένα blockchain, το οποίο επιπλέον παρέχει μία προγραμματιστική πλατφόρμα. Με λίγα λόγια, είναι ένα blockchain, επί του οποίου μπορούν να αναπτυχθούν αυτόνομα κομμάτια κώδικα, τα λεγόμενα έξυπνα συμβόλαια (smart contracts).
	
	Αυτά ενεργοποιούνται μόνο όταν δεχθούν από κάποιον λογαριασμό μία έγκυρη συναλλαγή και εκτελούν κώδικα με προκαθορισμένο τρόπο.
	
	Τώρα, συνδυάζοντας smart contracts με διεπαφές χρηστών μπορούμε να δημιουργήσουμε αποκεντρωμένες εφαρμογές (Decentralized Applications ή εν συντομία DApps). Οι εφαρμογές αυτές, κληρονομούν τις ιδιότητες του blockchain που αναλύσαμε προηγουμένως (π.χ. τη διαφάνεια, την αμεταβλητότητα και την επαληθευσιμότητα).
	
	Μία από τις χρήσεις τέτοιου είδους εφαρμογών είναι η παραγωγή token. Τα token είναι ας πούμε υπονομίσματα, δηλαδή, αντικείμενα ειδικού σκοπού με γενικά ασήμαντη αξία και μπορούν να έχουν διάφορες χρήσεις (είτε οικονομικές, είτε όχι). Μπορούμε δηλαδή να ορίσουμε ένα αυθαίρετο smart contract το οποίο να παράγει με προκαθορισμένο τρόπο ένα συγκεκριμένο σύνολο από token, να τα διαμοιράζει με συγκεκριμένο τρόπο κτλ. . Εμάς, όπως θα δούμε, μας ενδιαφέρει η χρήση τους ως αποδεικτικό δικαιώματος ψήφου σε ψηφοφορίες.
}
