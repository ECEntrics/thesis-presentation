\section{Στόχος της εργασίας}
\begin{frame}
	\frametitle{Στόχος της εργασίας}
	\vspace{-\baselineskip}

	Στόχος η δημιουργία μίας πρότυπης αυτόνομης, πλήρως αποκεντρωμένης κοινωνικής πλατφόρμας.
	\vspace{\baselineskip}

	Βασικά χαρακτηριστικά:
	\begin{itemize}
		\item Πρότυπη εφαρμογή (Proof of Concept)
		\item Πλήρης ελευθερία του λόγου
		\item Κυριότητα των χρηστών επί των δεδομένων τους
		\item Δυνατότητα διενέργειας αυθεντικών δημοκρατικών διαδικασιών
		\item Κρυπτογραφική εγγύηση αρτιότητας δεδομένων
		\item Αποκεντρωμένη αρχιτεκτονική
	\end{itemize}
\end{frame}

\note{
	Μέσα από την εργασία αυτή, στοχεύσαμε στην διερεύνηση του χώρου και στην πρότυπη υλοποίηση μίας λύσης που να αντιμετωπίζει/διευθετεί τα προβλήματα αυτά.

	Το επιθυμητό αποτέλεσμα ήταν η δημιουργία μίας ψηφιακής κοινότητας στην οποία ο ρόλος του εξυπηρετητή καθίσταται περιττός. Στην νέα πλατφόρμα αυτή, οδηγοί και διαχειριστές της συζήτησης είναι οι χρήστες οι οποίοι ως ηλεκτρονικοί πολίτες πρώτης κατηγορίας διατηρούν τον έλεγχο και την κυριότητα των δεδομένων που παράγουν. Βασικό χαρακτηριστικό της πλατφόρμας αποτελεί η δυνατότητα αυτοδιαχείρισης μέσα από αμεσοδημοκρατικές ψηφοφορίες.

	Έπειτα από διερεύνηση, συνειδητοποιήσαμε ότι οι στόχοι αυτοί είναι υλοποιήσιμοι και επικουρούνται μέσα από αποκεντρωμένες αρχιτεκτονικές σχεδίασης. Έτσι η αποκέντρωση αποτέλεσε πυλώνα της πλατφόρμας μας.

	Πιο απτά, η εφαρμογή που υλοποιήσαμε παρέχει τις βασικές δυνατότητες ενός forum, όπως ο διάλογος και η ψήφιση σε θέματα. Αργότερα θα δούμε αναλυτικότερα τις πλήρεις δυνατότητες και αδυναμίες της νέας πλατφόρμας.
}
